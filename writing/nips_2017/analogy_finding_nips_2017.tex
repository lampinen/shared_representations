\documentclass{article}

% if you need to pass options to natbib, use, e.g.:
% \PassOptionsToPackage{numbers, compress}{natbib}
% before loading nips_2017
%
% to avoid loading the natbib package, add option nonatbib:
% \usepackage[nonatbib]{nips_2017}

\usepackage{nips_2017}

% to compile a camera-ready version, add the [final] option, e.g.:
% \usepackage[final]{nips_2017}

\usepackage[utf8]{inputenc} % allow utf-8 input
\usepackage[T1]{fontenc}    % use 8-bit T1 fonts
\usepackage{hyperref}       % hyperlinks
\usepackage{url}            % simple URL typesetting
\usepackage{booktabs}       % professional-quality tables
\usepackage{amsfonts}       % blackboard math symbols
\usepackage{nicefrac}       % compact symbols for 1/2, etc.
\usepackage{microtype}      % microtypography
\usepackage{amsmath}

\title{From implicit to explicit: finding analogies with deep learning}

% The \author macro works with any number of authors. There are two
% commands used to separate the names and addresses of multiple
% authors: \And and \AND.
%
% Using \And between authors leaves it to LaTeX to determine where to
% break the lines. Using \AND forces a line break at that point. So,
% if LaTeX puts 3 of 4 authors names on the first line, and the last
% on the second line, try using \AND instead of \And before the third
% author name.

\author{
  Andrew Lampinen\thanks{http://web.stanford.edu/~lampinen`} \\
  Department of Psychology\\
  Stanford University\\
  Stanford, CA 94305 \\
  \texttt{lampinen@stanford.edu} \\
  %% examples of more authors
  %% \And
  %% Coauthor \\
  %% Affiliation \\
  %% Address \\
  %% \texttt{email} \\
  %% \AND
  %% Coauthor \\
  %% Affiliation \\
  %% Address \\
  %% \texttt{email} \\
  %% \And
  %% Coauthor \\
  %% Affiliation \\
  %% Address \\
  %% \texttt{email} \\
  %% \And
  %% Coauthor \\
  %% Affiliation \\
  %% Address \\
  %% \texttt{email} \\
}

\begin{document}
% \nipsfinalcopy is no longer used

\maketitle

\begin{abstract}
  The abstract paragraph should be indented \nicefrac{1}{2}~inch
  (3~picas) on both the left- and right-hand margins. Use 10~point
  type, with a vertical spacing (leading) of 11~points.  The word
  \textbf{Abstract} must be centered, bold, and in point size 12. Two
  line spaces precede the abstract. The abstract must be limited to
  one paragraph.
\end{abstract}

\section{Introduction}
Dual-process accounts of human reasoning suggest that we can reason both through implicit (neural network like) and explicit (symbolic, rule-based, and abstract) mechanisms \cite{Evans2003}. But how are these different systems connected? Can implicit knowledge become explicit? In this paper, we explore this issue within the domain of analogy. \par
Analogical reasoning has been considered an essential component of ``why we're so smart'' \cite{Gentner2003}. However, explicitly finding analogies is difficult -- even relatively successful algorithms like the structure mapping engine \cite{Falkenhainer1989} reduce to checking all possible isomorphisms between two sets in worst cases. Could implicit knowledge about the structure of the world help make explicit searches for analogies more targeted and efficient? \par
Previous work has shown that 
%%\subsubsection*{Acknowledgments}
%%
%%Use unnumbered third level headings for the acknowledgments. All
%%acknowledgments go at the end of the paper. Do not include
%%acknowledgments in the anonymized submission, only in the final paper.

\bibliographystyle{apacite}

\bibliography{shared_reps}

\end{document}
