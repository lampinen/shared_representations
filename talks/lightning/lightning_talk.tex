\documentclass{beamer}
\usepackage{pgfpages}
%\setbeameroption{show notes on second screen=left} %enable for notes
\usepackage{graphicx}
\usepackage{xcolor}
\usepackage{listings}
\usepackage{hyperref}
\lstset{language=python,frame=single}
\usepackage{verbatim}
%\usepackage{apacite}
\usepackage[longnamesfirst]{natbib}
\usepackage{subcaption}
\usepackage{amsmath}
\usepackage{relsize}
\usepackage{appendixnumberbeamer}
\usepackage{xparse}
\usepackage{multimedia}
\usepackage{xcolor}
\usepackage[normalem]{ulem}
\usepackage{tikz}
\usetikzlibrary{matrix,backgrounds}
\usetikzlibrary{positioning}
\usetikzlibrary{shapes,arrows}
\usetikzlibrary{positioning}

\tikzstyle{block} = [rectangle, draw, fill=red!20!blue!10, 
    text width=5em, text centered, rounded corners, minimum height=4em]
\tikzstyle{line} = [draw, line width=1.5pt, -latex']

\pgfdeclarelayer{myback}
\pgfsetlayers{myback,background,main}

\usetheme[numbering=fraction]{metropolis}
%%\AtBeginSection[]
%%{
%%  \begin{frame}
%%    \frametitle{Table of Contents}
%%    \tableofcontents[currentsection]
%%  \end{frame}
%%}

\begin{document}

\title{Toward a theory of transfer}
\author{Andrew Lampinen}
\date{Lightning Talk, Oct. 4th 2017}
\frame{\titlepage}

\begin{frame}{Transfer}
\begin{columns}<2>
\begin{column}{0.5\textwidth}
\end{column}
\begin{column}{0.5\textwidth}
\centering
\end{column}
\end{columns}
\end{frame}

\begin{frame}{The human condition}
\begin{figure}
\centering
\uncover<1->{%
\begin{subfigure}{0.5\textwidth}
\includegraphics[width=\textwidth]{figures/chess_cropped.jpg}
\end{subfigure}}%
\uncover<2->{%
\begin{subfigure}{0.5\textwidth}
\includegraphics[width=\textwidth]{figures/go.jpg}
\end{subfigure}}\\%
\uncover<3->{%
\begin{subfigure}{0.5\textwidth}
\includegraphics[width=\textwidth]{figures/math.jpg}
\end{subfigure}}%
\uncover<4->{%
\begin{subfigure}{0.5\textwidth}
\includegraphics[width=\textwidth]{figures/piano.jpg}
\end{subfigure}}%
\end{figure}
{\small \citealp*{Hansen2017}}
\end{frame}

\begin{frame}{Neural nets transfer too!}
\begin{figure}
\centering
\begin{tikzpicture}
\node (brain) {\includegraphics[width=0.25\textwidth]{figures/google_brain.jpg}}; 
\node [below = 1cm of brain] (input) {%
\only<2-3>{Ceci n'est pas une pipe.}
\only<5-6>{\includegraphics[width=0.25\textwidth]{figures/un_pipe.jpg}}
};
\node [above = 1cm of brain] (output) {%
\only<3>{This is not a pipe.}
\only<6>{This is a pipe.}
};

\only<2-3,5-6>{
\path [line] (input) -- (brain); 
}
\only<3,6>{
\path [line] (brain) -- (output); 
}
\end{tikzpicture}
\end{figure}
\end{frame}



\section{Toward a theory of transfer}

\begin{frame}{Isomorphic tasks for neural networks}

\end{frame}

\begin{frame}{Under what conditions?}

\end{frame}



\section{Wrapping up}

\begin{frame}{Conclusions \& future directions}

\end{frame}

%%\begin{frame}{Acknowledgements}
%%Thanks to:
%%\begin{itemize}
%%    \item Jay McClelland.
%%    \item The rest of the lab.
%%    \item The NSF, for funding.
%%    \item You, for listening. 
%%\end{itemize}
%%\end{frame}

\begin{frame}[standout]
Questions?
\end{frame}

\begin{frame}[allowframebreaks]
\bibliographystyle{plainnat}
{\tiny \bibliography{shared_reps}}
\end{frame}

\end{document}
